\section{Main results}
\label{sec:main_results}

We now present our main results for list-decodable and robust mixture learning defined in~\Cref{sec:prelims}. %\fy{The algorithm also provides guarantees for robust mixture learning}
In~\Cref{sec:ub_lb_sep}, we provide algorithmic upper bounds and information-theoretic lower bounds. For the special case of spherical Gaussian mixtures, we show in~\Cref{sec:comparison_prev_work} that we achieve optimality. 
Our results are constructive as we provide a meta-algorithm for which these bounds hold. 

As depicted in~\Cref{fig:alg_diagram}, our meta-algorithm (\Cref{alg:full_alg}) is a two-stage process. The outer stage (\Cref{alg:second_stage_pruning}) reduces the problem to mean estimation by leveraging the mixture structure and splitting the data into a small collection $\cT$ of sets $T$.
Each set $T \in \cT$ should (i) contain at most one inlier cluster (and few samples from other clusters) and (ii) the total number of outliers across all sets should be at most \(O(\e n)\).
We then run the inner stage (\Cref{alg:first_stage_high_level}) on sets \(T\), which outputs a mean estimate for the inlier cluster in $T$. First, a \(\cLD\) algorithm identifies the weight of the inlier cluster and returns the result of a \(\ckLD\) base learner with this weight. Then, if the weight is large, we improve the error via an \(\rme\) base learner. 
A careful filtering procedure in both stages achieves the significantly reduced list size and better error guarantees. 
We require the base learners to satisfy the following set of assumptions.

\begin{figure}
\centering
\begin{tikzpicture}[
    arrow/.style={-Latex, line width=1pt},
    smallblock/.style={draw, rectangle, rounded corners, line width=1pt, minimum height=1cm, text width=2.5cm, align=center, fill=gray!30},]
    % Outer Stage Box
    \node[rounded corners, draw=blue, line width=1pt, fill=blue!10, align=center, minimum width=6cm, minimum height=3.5cm, text width=7cm, text depth=15.0ex] (outer) at (0,0) {\textbf{Outer stage} (\Cref{alg:second_stage_pruning}): \\ creates collection \(\cT\) of sets \(T\)};

    % Inner Stage Box (container)
    \node[rounded corners, draw=magenta, line width=1pt, fill=magenta!10, minimum width=4cm, minimum height=2.0cm, text width=5cm, text depth=7.0ex] at ([shift={(3.8cm,-0.5cm)}]outer.west) (inner) {\textbf{Inner stage} (\Cref{alg:first_stage_high_level}): \\ for each \(T\) in \(\cT\):};

    % Inner Stage Box contents
    % Item 1
    \node[align=left] at ([shift={(0cm,-0.3cm)}]inner.center) (item1) {
           \begin{tabular}{@{}l@{}}
            1. Run algorithm for \(\cLD\)
        \end{tabular}
    };
    % Item 2
    \node[align=left] at ([shift={(-0.11cm,-0.7cm)}]inner.center) (item2) {
               \begin{tabular}{@{}l@{}}
            2. Improve errors if possible
        \end{tabular}
        };

    % Algorithm Boxes
    \node[smallblock, left=0.5cm and 1.5cm of inner.west, yshift=0.8cm] (algo1) {Base \(\ckLD\) Algorithm};
    \node[smallblock, below=0.2cm of algo1] (algo2) {Base \(\rme\) Algorithm};

    % Arrows to Algo Boxes
    \draw[arrow] (algo1.east) to [out=0,in=180]([yshift=-0.35cm]inner.west);
    \draw[arrow] (algo2.east) to [out=0,in=180]([yshift=-0.75cm]inner.west);
\end{tikzpicture}
\caption{Schematic of the meta-algorithm (\Cref{alg:full_alg}) underlying \Cref{thm:informal}}
\label{fig:alg_diagram}
\end{figure}



\begin{assumption}[Mean-estimation base learners for mixture learning]
\label{asm:algs}
Let \(t\) be an even integer and consider the corruption setting defined in~\Cref{def:cor-model}. Further, let the inlier distribution \(D(\mu^{\ast}) \in \cD\) where \(\cD\) is the 
the family of distributions satisfying~\Cref{def:bounded} for \(t\).
We assume that 
\begin{enumerate}[(a)]
    \item \label{asm:alg-ld} %for  
    for \(\alpha \in [\wmin, 1/3]\) in the \(\ckLD\) regime, there exists an algorithm $\ckLDA$ 
    that uses $N_{LD}(\alpha)$ samples and $T_{LD}(\alpha)$ time to output a list of size bounded by $1/\alpha^{O(1)}$  that with probability at least $1/2$ contains some $\muhat$ with $\norm{\muhat-\mu^*} \leq f(\alpha)$, where \(f\) is non-increasing.
    
    \item \label{asm:alg-rme} for $\alpha \in [1 - \e_{\rme}, 1]$, with \(0 \leq \e_{\rme} \leq 1/2 - 2\wmin^2 \) in the \(\rme\) regime, there exists an \(\rme\) algorithm $\cA_R$ 
    that uses $N_R(\alpha)$ samples and $T_{R}(\alpha)$ time to output with probability at least $1/2$ some $\muhat$ with $\norm{\muhat-\mu^*} \leq g(\alpha)$,
    where \(g\) is non-increasing. 
\end{enumerate}
\end{assumption}

Note that the sample and time-complexity functions such as \(N_{LD}\) and \(T_{LD}\), might depend on \(t\), for example growing as \(d^t\).
We emphasize that (i) the guarantees of our meta-algorithm depend on the guarantees of the base learners and
(ii) we only require the base learners to work in the well-studied setting with \emph{known} fraction of inliers.
Corollary~\ref{cor:gaussian} uses known base learners for Gaussian distributions achieving information-theoretically optimal error bounds.
There  also exists base learners for distributions beyond Gaussians, such as bounded covariance or log-concave distributions, see, e.g.~\cite{kothari2018robust}.
\subsection{Upper bounds for list-decodable mixture learning}
\label{sec:ub_lb_sep}
Key quantities that appear in our error bounds are the relative proportion of inliers $\tilde w_i$ and outliers  $\tilde \varepsilon_i$:
\begin{equation}
\label{eq:rel_weights}
\tilde w_i = \frac{w_i}{w_i + \varepsilon + \wmin^2} \quad \text{and} \quad \tilde \varepsilon_i = 1 - \tilde w_i.
\end{equation}
These quantities reflect that each set \(T\) in the inner stage contains at most one inlier cluster and a small (\(\lesssim \wmin^2\)) fraction of points from other inlier clusters.
We now present a simplified version of our main result in~\Cref{thm:informal}~(see~\Cref{thm:main-technical} for the detailed result) 
that allows for a more streamlined presentation of the results using 
the following `well-behavedness' of \(f\) and \(g\).
\begin{assumption}
    \label{asm:well-behaved} 
    Let \(f\), \(g\) be as defined in~\Cref{asm:algs}. For some \(C > 0\), we assume (i) \(\e_{\rme} \geq 0.01\), (ii) $\forall x \in (0, 1/3]$, $f(x/2) \leq C f(x)$, and (iii) $\forall x \in [0.99, 1]$, $g(x - (1-x)^2) \leq C g(x)$.
\end{assumption}
We are now ready to state the main result of the paper.
\begin{theorem}
\label{thm:informal}
    Let $d, k \in \N_+$, $\wmin \in (0, 1/2]$, and $t$ be an even integer. 
Let $\cX$ be a $d$-dimensional mixture distribution following~\Cref{eq:gen_model}.
Let \(\ckLDA\) and \(\cA_R\) satisfy~\Cref{asm:algs,asm:well-behaved} for some even \(t\). 
Further, suppose that $\norm{\mu_i - \mu_j} \gtrsim \sqrt{t}(1/\wmin)^{4/t} + f(\wmin)$ for all $i \neq j \in [k]$.

Then there exists an algorithm that, given ${\poly(d, 1/\wmin) \cdot (N_{LD}(\wmin) + N_R(\wmin))}$ i.i.d. samples from $\cX$ as well as $d$, $k$, $\wmin$, and $t$, runs in time ${\poly(d, 1/\wmin) \cdot (T_{LD}(\wmin) + T_R(\wmin))}$ and with probability at least $1-\wmin^{O(1)}$ outputs a list $L$ of size ${|L| \leq k + O(\epsilon/\wmin)}$ where, for each $i \in [k]$, there exists $\hat\mu \in L$ such that
    \[\norm{\muhat - \mu_i} = \bigO{\min_{1\le t'\le t}\sqrt{t'} (1 / \tilde w_i)^{1/t'}+f(\min(\tilde{w}_i, 1/3))}.\]
    If the relative weight of the \(i\)-th cluster is large, i.e., \(\tilde \e_i \leq 0.001\), then the error is further bounded by
    \[\norm{\muhat - \mu_i} = \bigO{g(\tilde{w}_i)}.\]
\end{theorem}
The proof together with a more general statement,~\Cref{thm:main-technical}, can be found in~\Cref{sec:main_result_appendix}.

Note that for a mixture setting with $k\geq 2$, the assumption \(\wmin \leq 1 / k \leq 1/2 \) is automatically fulfilled.
Also, for large weights $\tilde w_i$ such that $\log(1 / \tilde w_i) \ll t$, the $t'$ that minimizes $\sqrt{t'} (1 / \tilde w_i)^{1/t'}$ is smaller than $t$, and for small weights the minimizer is $t'=t$.
