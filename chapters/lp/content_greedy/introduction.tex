The forthcoming results are valid under the conditions listed below, which will be assumed to hold throughout.
\begin{assumption}\label{asmpt:A_np_m}We assume that

% \begin{enumerate}[ref={\theassumption.\arabic*}]
\begin{enumerate}
    \item Each element $j\in \mathcal{X}$ is assigned to any subset $\mathcal{S}_i$, $i \in [m]$ with probability $p \equiv p(n)$, independently. That is, $\A \in \{0, 1\}^{m \times n}$ is such that $A_{ij} \overset{\text{iid}}{\sim} \mathrm{Bernoulli}(p)$; 
    \item $n$ is intended to be large but finite;
    \item \label{asmpt:A_np_m.c} \(m \equiv m(n) = \mathrm{poly}(n)\), i.e. $\exists c, C > 0$, such that $c n^c \leq m \leq Cn^C$ for $n$ large enough; 
    \item 
\label{asmpt:A_np_m.d} There exist $\delta \in (0,1)$, such that \(p \equiv p(n)\) satisfies \(1/n^{\delta} \leq p \leq 1/2\), for all \(n\) large enough.
\end{enumerate}
\end{assumption}

\noindent
Note that in Assumption~\hyperref[asmpt:A_np_m.c]{1.1.2.3}, the upper bound is chosen to avoid trivial solutions w.h.p. which arise, for example, in the setting where the number of sets grows exponentially in the cardinality of $\mathcal{X}$.  
In addition, Assumption~\hyperref[asmpt:A_np_m.d]{1.1.2.4} is by no means restrictive, since one can show that for $m = \text{poly}(n)$ and \(np \ll \log n\), we have that $\A$ contains an all-zero row w.h.p., yielding an infeasible solution for $\ip$. 
The requirement \(p \leq 1/2\) is chosen for technical convenience and can be relaxed to any constant \(p\), encompassing the regime in~\cite{iliopoulos2021group}.\\

\noindent
Our contributions stem from the study of the size of the inclusion sets $I_j := \set{i \in [m]\: : \: j \in \mathcal{S}_i}$, for $j \in [n]$, which in the \(\mvch\) formulation of the problem at hand correspond to the set of hyperedges incident to any given vertex. 
The key quantity under study is the average inclusion set size, that is $\E |I_j| = mp$, for all $j$, under the present distributional assumptions. 
This quantity exhibits two separate regimes of interest, referred to as the \emph{sparse}, $mp \ll \log{n}$, and \emph{dense}, $mp \gg \log{n}$, regimes. 
These, in turn, determine the size of the maximum inclusion set, or maximum degree, $\dmax := \max_{j \in [n]} |I_j|$. 
We characterize the integrality gap behaviour up to multiplicative constants and analyse Lovász's $\greedy$ algorithm \cite{lovasz1975ratio} in these two regimes w.h.p as $n \to \infty$. 
We do this by proving the success of a simple greedy heuristic, the \bgreedy algorithm (Algorithm \ref{alg:block_greedy}). 
Throughout, we use the notation $\text{val}_{\mathtt{Gr}}$, $\valbgr$ to denote the size of the hitting set returned by $\greedy$ and $\bgreedy$ respectively. Below we provide an informal description of the main results which hold with high probability, where $A(n) \sim B(n)$ denotes that $c A(n) \leq B(n) \leq C A(n)$ for large enough $n$ and for some constants $c, C > 0$: 
\\
% \noindent

\section{Main contributions}\label{sec:intro}



%%%%%%%%%%%%%%%%%%%%%%%%%%%%%%%%%%%%%%%%%%%%%%%%%%%%%%%
% \begin{figure}
%     \centering
%   \begin{tikzpicture}
%     \begin{axis}[ymin = 0, ymax = 0.45, axis lines=middle, height=8cm, width=14cm,
%     legend style={
%       at={(1,1)}
%       },
%     axis line style={->},
%     x label style={at={(axis description cs:0.96,-0.14)},anchor=south},
%     y label style={at={(axis description cs:-0.05,0.9)},anchor=south},
%     xlabel={\LARGE $m(n)$},
%     ylabel={\LARGE $p(n)$},
%     xtick={1.58, 3.98, 10, 25.11},
%     xticklabels={\large $n^{0.1}$,
%                 \large $n^{0.3}$,
%                 \large $n^{0.5}$, 
%                 \large $n^{0.7}$},
%     ytick={0.01,0.039,0.1,0.25,0.63},
%     yticklabels={\large $1 / n^{0.9}$,
%                 \large $1 / n^{0.7}$,
%                 \large $1 / n^{0.5}$,
%                 \large $1 / n^{0.3}$,
%                 \large $1 / n^{0.1}$},]
%         \addplot[thick, color=red, smooth, name path = inv, domain = 1:40, line width=2pt] {1.5/x};
%         \addplot[name path = floor, draw = none] coordinates {(1,0) (40,0)};
%         \addplot[fill=red, 
%         fill opacity=0.2] fill between[of = inv and floor];
%         \path[name path=axis] (axis cs:0,40) -- (axis cs:40,40);
%         \addplot[fill = blue, fill opacity=0.3] fill between[of= inv and axis];
%         \legend{\large $ mp \sim \log n$}
%     \node[below, text=red!30!black,font=\bfseries\Large] at (12, 0.055) {\valbgr \(\sim\) \valip \(\sim\) \vallp \(\sim m / \dmax\)};
%     \node[above, text=blue!80!black, font=\bfseries\Large] at (23,0.2) {\valbgr \(\sim\) \valip \(\sim \log\left(\frac{mp}{\log n}\right) / p \)};
%     \node[above, text=blue!80!black, font=\bfseries\Large] at (18,0.17) {\vallp \(\sim 1/p\)};
% \end{axis}
% \end{tikzpicture}
% \caption{Transition between the sparse and the dense regime for different values of the average inclusion set size $mp$.}
%     \label{fig:phase_diagram}
% \end{figure}

\noindent
The results above are also depicted in Figure \ref{fig:phase_diagram}, and the formal statements are given in~\Cref{cor:ipgap_collected} and Theorem~\ref{thm:greedy}. The rest of the chapter is organized as follows. 
In Section \ref{sec:notation}, we present relevant notation. In Section \ref{sec:related work}, we outline and discuss related literature. 
In Section \ref{sec:bounds_LP_IP}, we prove a number of preliminary results that will be instrumental in developing the core arguments. 
Subsequently, in Section~\ref{sec:algo_sol}, we delve into the algorithmic aspects of the problem at hand by first providing guarantees for a simple algorithm, \bgreedy. We then analyse \greedy by means of a reduction. 
We conclude in Section \ref{sec:discussion} by summarizing the results and offering indications for future work. We defer the proofs of more technical results to the appendix, in order to streamline the presentation for the reader's convenience.

\section{Notation and conventions}\label{sec:notation}
For integers $k \in \mathbb{N}$, we write $[k]:=\set{1,...,k}$. 
We denote vectors, matrices by bold-faced Roman letters $\x, \A \in \mathbb{R}^k, \mathbb{R}^{k \times k}$, respectively, for some $k \in \mathbb{N}$. 
Define the \emph{inclusion set} of an element, or node, $j \in [n]$ as $I_j = \left\{i \in [m] : j \in \mathcal{S}_i \right\}$. 
We denote the $\ell_1$ norm of the $j$-th column of $\A$ by $X_j$, $j\in [n]$, noting that $X_j = |I_j|$ and $X_1, \ldots X_n \overset{\text{iid}}{\sim} \mathrm{Binomial}(m, p)$. 
In addition, we let $\dmax \equiv \dmax(X_1, \ldots, X_n) := \max_{i \in [n]} X_i$. 
We use $\E, \text{Var}$ to denote expectation and variance, respectively. 
By $\lesssim$, $\gtrsim$ we denote inequalities up to multiplicative constants. 
We let $A \sim B$ denote that $A \lesssim B \lesssim A$ for large enough $n$. 
We let $\log$ denote the natural logarithm. 
For possibly random functions \(f(n), g(n)\), we let \(\{f \lesssim g\}\) denote a sequence of events \(\{f(n) \leq A g(n)\}\) for some constant \(A > 0\) independent of \(n\). 
Consequently, \(\Pr(f \lesssim g)\) is viewed as a function of \(n\). For deterministic functions $h(n), w(n)$, we let $h \ll w, h \gg w$ denote that $h/w \to 0, w/h \to 0$ respectively, as $n \to \infty$. 
The notation for other inequalities is defined analogously. 
We say that a sequence of events $\set{A_n}$ holds \emph{with high probability} (w.h.p.) with respect to a probability measure $\Pr$ if there exists a constant $c > 0$, independent of $n$, such that $\Pr(A_n) \geq 1-n^{-c}$, for large enough values of $n$.